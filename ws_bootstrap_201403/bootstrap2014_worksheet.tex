\documentclass[10pt, a4paper]{article}

\usepackage{amsmath}
\usepackage{amssymb}
\usepackage{fullpage}
\usepackage{algorithm2e}
\usepackage{graphicx}
\usepackage{wrapfig}



\newcounter{wssection}
\newcounter{wsexercise}[wssection]


\newcommand{\worksheetsection}[1]{
\vspace{10mm}
\stepcounter{wssection}
\noindent \Large \textbf{\thewssection. #1} \normalsize
\vspace{3mm}
}


\newcommand{\worksheetexercise}{
\stepcounter{wsexercise}
\vspace{5mm} \noindent \textbf{Exercise \thewssection.\thewsexercise \;}
}


\title{Dublin R Workshop on Bootstrapping}
\author{Mick Cooney\\mickcooney@gmail.com}
\date{March 26, 2014}


\begin{document}

\maketitle

\worksheetsection{Introducing the Bootstrap}

\worksheetexercise Using the code in file \texttt{basic\_bootstrap.R},
run a basic bootstrap on the mean of the air conditioning data.

\worksheetexercise Use the \texttt{boot.ci()} function to estimate the
95\% confidence interval for bootstrapped mean.

\worksheetexercise Generate random normal data using \texttt{rnorm()},
and summarise to the data.

\worksheetexercise Run the bootstrap on the data to get some estimates
on the mean of the data.



\worksheetsection{Using the Bootstrap with Regression}

\worksheetexercise Open up the code in \texttt{lm\_bootstrap.R} and
see what it is doing. Make sure you understand it.

\worksheetexercise Run the code and examine the output.

\worksheetexercise Interpret the output of the bootstrap, and
visualise your output.


\worksheetsection{Using the Bootstrap with Portfolio Optimisation}

\worksheetexercise Using the data in \texttt{equity\_returns.rds}, and
the function \texttt{portfolio.optim()} in the \emph{tseries}
package, calculate the optimum portfolio weights for the four equities
given.

\worksheetexercise Using the code in \texttt{portfolio\_bootstrap.R},
run the bootstrap on the portfolio optimisation problem. Investigate
the output. Use 120 days for each bootstrap.

\worksheetexercise Investigate the effect of bootstrap sample size on
the output of the bootstrap.


\worksheetsection{Using the Bootstrap with Time-series data}


\worksheetexercise Load the lynx data set and get an estimate for the
order of the time-series using the \texttt{ar()} function.

\worksheetexercise Run the bootstrap code in
\texttt{ts\_bootstrap.R}. Examine the code to understand what it is
doing.

\worksheetexercise Examine the output of the bootstrap and interpret
its output.



\end{document}
